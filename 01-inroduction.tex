\section{Introduction}\label{sec:introduction}
The role of JavaScript drastically changed in recent years. From an unpopular language used to add dynamic effects to web pages to one widely used with a strong and growing community. It emerged from a browser-only language to a general-purpose language used to write web-, desktop-, mobile-, and server-applications. The new use cases come along with new requirements demanding for more computational resources that can no longer be provided by a single threaded environment without affecting the user experience. The W3C responded to this new requirements with a first draft for the Web Worker API in 2009 providing the infrastructure for creating multithreaded applications~\cite{IanHickson2015}. However, the web worker API requires that the code for web workers is located in a separate file resulting in an increased complexity to the build procedure. The Communication between the main thread and web workers --- and between web workers --- is only messaging based. The messaging based programing model often does not fit well into the existing application design. 

Besides that the programing model from Web Workers differs from the normal JavaScript programing model it is also not universal across platforms. NodeJS provides the child-process API that allows to spawn new child processes which allows to simulate a multithreaded application. However, the processes have no access to a shared memory. I believe that platform dependent APIs together with the inherent complexity of the provided low level apis are the main reason for the low adaption and low exploitation of the available computation resources.