\section{Programming Model}\label{sec:programming-model}
The programming model of Parallel.es motivates the programmer to perform time-intensive computations concurrently in background tasks without blocking the main thread. 

\subsection{Background Task}
A \textit{background task} --- further referred to as task --- represents an operation that is asynchronously started in the main thread and is executed in a background thread. \Cref{fig:fibonacci-implementation} shows an example of a task computing the Fibonacci number of $100$ in a background thread. The \javascriptinline/run/ method creates a new task implemented by the passed function (line \ref{code:parallel-run}) that is invoked in a background thread with the provided arguments. Since the returned object implements the promise interface~\cite[Section 18.3.18]{ecmaScript2015} a \javascriptinline/then/ and an optional \texttt{catch} callback can be registered. The \javascriptinline/then/ callback is invoked with the task-result if the computation is successful and the \texttt{catch} callback otherwise. These callbacks are executed on the main thread and allow the retrieval of the result or error. 


\begin{listing}
	\begin{javascriptcode}
function fib(num) {
	if (num <= 2) {
    	return 1;
	}

	return fib(num - 1) + fib(num - 2);
}
        
parallel.run(fib, 100) |$\label{code:parallel-run}$|
	.catch(error => console.error(error))
	.then(result => console.log(result));	
	\end{javascriptcode}

	\caption{Fibonacci Implementation}
	\label{fig:fibonacci-implementation}
\end{listing}

The JavaScript functions used to implement a task have to comply with the following discussed restrictions. References to non-read-only variables from the outer scope of a task function are prohibited\footnote{The special identifiers \javascriptinline/this/ and \javascriptinline/super/ are treated as references to non-read-only-variables from the outer scope, and their usage inside of a task function is, therefore, prohibited. This restriction also implies that an arrow function used as task function is semantically equal to a function expression.} as well as references to functions not resolvable by static scoping, e.g. a function passed as an argument. Moreover, the values of referenced variables, the arguments passed to the task function, and the value returned by the task function are passed \enquote{by value} and have to be structured cloneable~\cite[Section 2.9.4]{WHATWG2016}, e.g. this is not the case for DOM-Elements and Errors. The structured cloning is enforced by the underlying, messaging-based standards and can not be abstracted by the runtime system. The passing of not cloneable values results in runtime errors. On the contrary, illegal references to non-read-only variables from the inside of a task function are detected by the transpiler.

In contrast to standard JavaScript functions, the global context of the background thread executing the task function differs from the global context in which the function is defined. Therefore, the APIs accessible in background threads may vary from the one offered in the main thread, e.g. the DOM API is not accessible to web workers. Furthermore, changes made to the global context of one thread are not visible to the other threads since each thread has its distinct global context. Therefore, the global context cannot be used to store shared state. These are no significant limitations for task functions since they, in general, perform compute-intensive, but side effect free, operations only depending on local data.

Tasks are isolated from one another since threads share no variables, and every thread executes one task at a time. However, the tasks executed in a specific background thread share the same global context. It is, therefore, possible that two tasks affect each other if they access and modify the same global context. Modification to the global context are not prevented but are strongly discouraged as changes are only thread-local and may introduce memory leaks. The next section describes the reactive API that simplifies the parallelization of data-stream-based operations.

\subsection{Reactive API}
The runtime system further offers a reactive API~\cite{Meijer2012}. The goal of it is to provide a well-known and understood API that uses the available computation resources without any further doing of the programmer. The API is inspired by the commonly used underscore~\cite{underscorejs} and lodash~\cite{lodash} libraries and motivates the programmer to define the computations as operations on data streams. This allows the runtime system to take care of splitting the work into several sub-tasks and aggregating the task-results into the end-result. The reactive API uses the infrastructure provided by the low-level API and therefore, the same programming model applies.


 An implementation of the Mandelbrot computation using the reactive API of Parallel.es is shown in \cref{code:mandelbrot-parallel.es}. It differs only slightly from the sequential, lodash~\cite{lodash} based implementation shown in \cref{fig:mandelbrot-sync}. The differences between the implementations are highlighted in gray. This alikeness of the APIs facilitates a fast learning curve and simplifies transitioning of existing code. 
 
 The \javascriptinline/range/ method (line \ref{code:parallel-es-definition-start}) defines the data stream to process. It creates a data stream containing the values from 0 up to the image height (exclusive) that is transformed by mapping (\javascriptinline/map/ on line \ref{code:parallel-es-map}) each element to an output element that is computed by the \javascriptinline/computeMandelbrotLine/ function (line \ref{code:parallel-es-operation}). The \javascriptinline/computeMandelbrotLine/ function --- that is executed in a background thread --- has access to the current array element and the read-only variables from its outer scope. It can further make use of the \javascriptinline/computePixel/ (line \ref{code:mandelbrot-compute-pixel}) function defined in its outer scope or functions imported from other modules. The computation is started using the \javascriptinline/then/ method (line \ref{code:parallel-es-start}) that registers a callback. The \textit{then} callback is executed in the main thread and is invoked with a single array containing the joined lines of the Mandelbrot if the computation succeeds. An optional error callback can be defined that is invoked in case the execution fails. The API further allows retrieving sub-results by registering the \textit{next} callback using the \javascriptinline/subscribe/ method (line \ref{code:parallel-es-subscribe}). The \textit{next} callback is invoked whenever a task has completed and is passed the lines computed by this task, the index of this task, and the number of lines computed by each task. The sub-results can be used to show a progress update, e.g. drawing already computed lines of the Mandelbrot instead of waiting until all lines have been computed. The \textit{next} callback is invoked in the main thread and the order of task completion. 
 
 The following section describes the functioning of the runtime system orchestrating the background tasks.
 
\begin{listing}
	\begin{javascriptcode*}{highlightlines={17-21}}
const imageWidth = 10000;
const imageHeight = 10000;

function computePixel(x, y) { |$\label{code:mandelbrot-compute-pixel}$|
	// ...
	return n;
}

function computeMandelbrotLine(y) { |$\label{code:parallel-es-operation}$|
	const line = new Uint8ClampedArray(imageWidth * 4); |$\label{code:mandelbrot-reference1-imageWidth}$|
	for (let x = 0; x < imageWidth; ++x) {
		line[x * 4] = computePixel(x, y);
	}
	return line;
}

parallel
	.range(imageHeight) |$\label{code:parallel-es-definition-start}$|
	.map(computeMandelbrotLine)	 |$\label{code:parallel-es-map}$|
	.subscribe((subResult, index, batchSize) => ...) |$\label{code:parallel-es-subscribe}$|
	.then(result => console.log(result)); |$\label{code:parallel-es-start}$|
	\end{javascriptcode*}
	
	\caption{Mandelbrot Implementation in Parallel.es}
	\label{code:mandelbrot-parallel.es}
\end{listing}


\begin{listing}
	
	\begin{javascriptcode}
const imageWidth = 10000;
const imageHeight = 10000;

function computePixel(x, y) {
	// ...
	return n;
}

function computeMandelbrotLine(y) {
	const line = new Uint8ClampedArray(imageWidth * 4);
	for (let x = 0; x < imageWidth; ++x) {
		line[x * 4] = computePixel(x, y);
	}
	return line;
}

const result = _.chain()
	.range(imageHeight)
	.map(computeMandelbrotLine)
	.value();
	
console.log(result);
\end{javascriptcode}
\caption{Sequentiall, Lodash~\cite{lodash} based Mandelbrot Implementation}
\label{fig:mandelbrot-sync}
\end{listing}