\section{Programing Model}\label{sec:programing-model}
The programing model of Parallel.es motivates the programer to perform time intensive computations asynchronous, and potentially concurrently, in background-tasks. 

\subsection{Background Task}
A \textit{background task} --- further referred to as task --- represents a single asynchronous operation executed on a background-thread and is implemented by a JavaScript function. \Cref{fig:fibonacci-implementation} shows an example that computes the fibonacci number for the value 100 in a task and logs the result to the console. The task is started using the \javascriptinline/run/ method by passing the function to execute together with the arguments for the function call (line \ref{code:parallel-run}). The returned object implements the promise interface~\cite[Section 18.3.18]{ecmaScript2015} allowing to register a \javascriptinline/then/ callback that is invoked with the result if the computation was successful and an error handler (\javascriptinline/catch/) that is triggered otherwise. The callbacks are executed on the main-thread and allow retrieval of the result. 


\begin{listing}
	\begin{javascriptcode*}{highlightlines={9-11}}
function fib(num) {
	if (num <= 2) {
    	return 1;
	}

	return fib(num - 1) + fib(num - 2);
}
        
parallel.run(fib, 100) |$\label{code:parallel-run}$|
	.catch(error => console.error(error))
	.then(result => console.log(result));	
	\end{javascriptcode*}

	\caption{Fibonacci Implementation}
	\label{fig:fibonacci-implementation}
\end{listing}

The valid task-functions are only a subset of all JavaScript functions since not all JavaScript features are allowed to be used. This is rooted in the fact that neither web workers nor child processes have a shared memory\footnote{This might change with the SharedMemory~\cite{Ecma2016} standard that is currently a draft in stage 2.} that can be used to store data accessed by different threads. It is therefore disallowed that a task-function references any variables or functions --- including imports from external files --- from the outer scope\footnote{The special identifiers \javascriptinline/this/ and \javascriptinline/super/ are treaded equally to other identifiers referring to variables from the outer scope and their usage inside of a task-function is therefore prohibited. This also implies that an arrow function used as task-function is semantically equal to a function expression.}. Data needed by a task function needs to be explicitly passed when the task is started and is serialized using the structured clone algorithm~\cite[Section 2.9.4]{WHATWG2016}. This enforces a pass-by-value semantic and requires that the passed data is serializable according to the structured clone standard, e.g. Functions, DOM-Elements and errors are not cloneable. The same applies for the result returned by the task. 

The global context of a task differs from the global context the function is defined it. Changes made in the main-thread or inside of a single task are not reflected across the threads. Furthermore, the api exposed by the environment may differ, e.g. the DOM api is not exposed inside of web workers. This does not add strong limitations to task-functions since task functions mainly are used to perform compute intensive operations that depend on local data only.

Parallel tasks are isolated from one another since each tasks operates on its own copy of the variables and every thread only executes on task at the time. However, the global state of a thread is shared among all tasks executed on this thread instances. Tasks can access and modify the global state. This is strongly discouraged as changes are only thread local --- and therefore not replicated between threads --- and may introduce memory leaks. 

\subsection{Reactive API}
The runtime system further offers a reactive api. This api is inspired by the commonly used underscore and lodash libraries and motivates the programer to define the computations as operations on data streams. The runtime system takes care of splitting the work into several sub tasks and aggregating the sub results to the end result. The created tasks perform all operations on a subset of the input values and are potentially executed concurrently if enough resources are available. The goal of this api is to provide a well known and understood api that uses the available computation resources without any further actions needed by the programer. This simplifies transforming synchronous code into parallelized. The reactive api uses the infrastructure provided by the low level api. Therefore, the same programing model applies.


 An implementation of the Mandelbrot computation using the reactive api of Paralle.es is shown in \cref{code:mandelbrot-parallel.es}. The implementation differs only slightly from the synchronous implementation shown in \cref{fig:mandelbrot-sync}. The \javascriptinline/range/ method (line \ref{code:parallel-es-definition-start}) defines the data stream to process. It creates a data stream containing the values from 0 up to the image height. The input stream is transformed by mapping (\javascriptinline/map/ on line \ref{code:parallel-es-map}) each element from the input stream to an output element that is computed by the \javascriptinline/computeMandelbrotLine/ function (line \ref{code:parallel-es-operation}). The \javascriptinline/computeMandelbrotLine/ function --- that is executed in a background-thread --- has access to the current array element and as well to the \javascriptinline/environment/ object that was passed in the \javascriptinline/inEnvironment/ call. The computation is started using the \javascriptinline/then/ method (line \ref{code:parallel-es-start}) that registers a callback to invoke in case the computation succeeds --- an optional error callback can be defined as well. The then callback is executed in the master-thread.  No additional build step is required.

\begin{listing}
	
	\begin{javascriptcode}
function computeMandelbrotLine(y, options) {
	const arraySize = options.imageWidth * 4;
	const line = new Uint8ClampedArray(arraySize);
	// compute...
	return line;
}

const range = _.range(options.imageHeight);
const result = range.map(
	y => computeMandelbrotLine(y, options)
);
console.log(result);
\end{javascriptcode}
\caption{Synchronous Mandelbrot Implementation}
\label{fig:mandelbrot-sync}
\end{listing}

\begin{listing}
	\begin{javascriptcode*}{highlightlines={2, 8-12}}
function computeMandelbrotLine(y, environment) { |$\label{code:parallel-es-operation}$|
	const arrazSize = environment.imageWidth * 4;
	const line = new Uint8ClampedArray(arraySize);
	// compute...
	return line;
}

parallel      |$\label{code:parallel-es-definition-start}$|
	.range(options.imageHeight)
	.inEnvironment(options)
	.map(computeMandelbrotLine) |$\label{code:parallel-es-map}$|
	.then(result => console.log(result); |$\label{code:parallel-es-start}$|
	\end{javascriptcode*}
	
	\caption{Mandelbrot Implementation in parallel.es}
	\label{code:mandelbrot-parallel.es}
\end{listing}


