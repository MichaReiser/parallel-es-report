\section{Runtime System}\label{sec:runtime-system}
The runtime system of Parlallel.es consists of two parts: The slaves running in background-threads executing the tasks and the public api in the main-thread that forms the facade and acts as the master for the slaves. Applications are using the facade provided by the master to run functions in a background-thread. The master is responsible for spawning the slaves and distributing the work onto these. The runtime system uses a thread pool that queues task in the case no idle background-thread is available. The default thread pool uses a FIFO queue and the number of background-threads is limited to the number of logical processors provided by the hardware\footnote{The number of logical processors can be determined using \javascriptinline/navigator.hardwareConcurrency/. The implementation assumes the hardware has four logical processors if the api is not supported by the used browser.}. The next section describes how the runtime system processes a single task. 

\subsection{Task Roundtrip}
The steps needed to process a single task are shown in \cref{fig:runtime-system}. The application passes the task-function together with the arguments to the facade that acts as master (1). The created task is enqueued in the thread pool and executed on the first slave that get available. The master transmits the serialized representation of the function call --- consisting of a unique id identifiying the function to call and the arguments to pass to the function --- to the slave (2). The slave performs a lookup in the function cache to retrieve the function with the given id (3). If the function is executed the first time on this slave, than the function definition is requested from the master (4, 5), instantiated as a function and registered in the function cache (6). Otherwise, the deserialized function is returned by the function cache. The slave calls the deserialized function with the provided arguments (7) and returns the (structured cloned) result back to the master (8). The master invokes the success handler in the main-thread to pass the result to the application (9). 

\begin{figure*}
	\centering
	\includegraphics[width=0.8\textwidth]{runtime-system.pdf}

	\caption{Parallel.es Runtime System}
	\label{fig:runtime-system}
\end{figure*}

The caching of the function definitions on the slave have the advantage that performed JIT-optimizations are not thrown away if a task has completed. It is believed that the gain of reusing the JIT-Optimized function outweighs the additional costs caused by the function lookup and additional roundtrip for function retrieval. Especially for frequent but short running tasks. 


\subsection{Limitations}
The current implementation supports the most essential features. However, it misses support for asynchronous task operations and the NodeJS environment. There are no technical reasons for not supporting either of these features. Adding support for NodeJS requires a structured clone polyfill to have the exact same behavior in NodeJS as it is in the browser.

A further limitation is that a task can not start other tasks. An efficient implementation to support recursive tasks requires a communication channel between all web workers that allows to start a task from a background-thread on another, idle background-thread without an additional roundtrip over the main-thread. However, Web Workers only allow to have a single communication channel that is between the thread that has started the Web Worker and the thread running the Web Worker. Multiple channels are supported by Shared Workers that lack support in older browsers~\cite[section 4.6.4]{w3cWebWorker}.