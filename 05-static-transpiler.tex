\section{Transpiler}\label{sec:transpiler}
The absence of a shared memory that is accessible by all threads\footnote{This might change with the SharedMemory~\cite{Ecma2016} standard that is currently a draft in stage 2.} and allows to store shared variables requires an explicit passing of the variables referenced by the task function to the background thread executing the task. The transpiler covers this explicit passing of the variables by rewriting the program code. It extracts the task functions from the code run in the main thread and adds it to the file loaded by the slaves. The transpiler further adds the imports referenced by the task function and pre-registers it in the slave's function cache. The transpiler is based on top of webpack\footnote{A JavaScript module bundler.}~\cite{webpack} and Babel\footnote{A framework for Transforming JavaScript code.}~\cite{babel}. The use of the transpiler is optional if task functions are not referencing symbols from their outer scope. 


The functioning of the transpiler is following explained by using the Mandelbrot implementation shown in \cref{code:mandelbrot-parallel.es}. The transpiler rewrites the program code to make the variable \javascriptinline/imageWidth/ (line \ref{code:mandelbrot-reference1-imageWidth}) and the function \javascriptinline/computePixel/ (line \ref{code:mandelbrot-compute-pixel}) available to the task function. \Cref{fig:mandelbrot-transpiled} shows the transpiled Mandelbrot implementation in which the changes made by the transpiler are highlighted in gray. The transpiler creates the new function \javascriptinline/_environmentExtractor/ (line \ref{code:mandelbrot-environment-extractor}) that returns an object containing the values of all referenced variables and inserts it above the definition of the task function. This function is used to extract the value of the referenced variable \javascriptinline/imageWidth/ in the master thread. The object returned by the \javascriptinline/_environmentExtractor/ function is made available to the task function by setting it as environment using the \javascriptinline/inEnvironment/ method (line \ref{code:mandelbrot-environment-call}). The runtime system passes the environment object as the last argument when invoking the task function. The transpiler further replaces the task definition by a unique function-id (lines \ref{code:mandelbrot-function-id-start}-\ref{code:mandelbrot-function-id-end})\footnote{The transpiler does not remove the task function from the code run in the main thread since it might be referenced elsewhere. Removing unreferenced functions is left to minifiers as proofing a function to be truly unused is non-trivial.}. 

\begin{listing}
	\begin{javascriptcode*}{highlightlines={9-13, 25-29}}
const imageWidth = 10000;
const imageHeight = 10000;

function computePixel(x, y) {
	// ...
	return n;
}

function _environmentExtractor() { |$\label{code:mandelbrot-environment-extractor}$|
	return {
		imageWidth: imageWidth
	};
}

function computeMandelbrotLine(y) {
	const line = new Uint8ClampedArray(imageWidth * 4);
	for (let x = 0; x < imageWidth; ++x) {
		line[x * 4] = computePixel(x, y);
	}
	return line;
}

parallel
	.range(imageHeight)
	.inEnvironment(_environmentExtractor()) |$\label{code:mandelbrot-environment-call}$|
	.map({ |$\label{code:mandelbrot-function-id-start}$|
		identifier: "static:_entrycomputeMandelbrotLine",
		_______isFunctionId: true
	}) |$\label{code:mandelbrot-function-id-end}$|
	.then(result => console.log(result));
\end{javascriptcode*}
\caption{Transpiled Mandelbrot Implementation}
\label{fig:mandelbrot-transpiled}
\end{listing}


\Cref{fig:transpiled-mandelbrot-slave} shows the code inserted by the transpiler into the script run by the slaves. The transpiler injects the code of the task function (lines \ref{code:slave-task function-start}-\ref{code:slave-task function-end}) and the referenced \javascriptinline/computePixel/ function (lines \ref{code:slave-compute-pixel-start}-\ref{code:slave-compute-pixel-end})\footnote{The transpiler wraps the functions of each module in the code run on the slaves with an immediately invoked function expression to isolate the functions of one module from the others and avoid naming conflicts.}. Further, an \textit{entry}-function (lines \ref{code:slave-entry-function-start}-\ref{code:slave-entry-function-end}) is generated that initializes the \javascriptinline/imageWidth/ variable (line \ref{code:slave-image-width}) with the value stored in the environment --- that contains the values of the variables from the main thread --- and calls the actual task function. The entry function is registered in the function cache (lines \ref{code:slave-register-function-start}-\ref{code:slave-register-function-end}) using the same function-id as utilized in the master thread. This pre-registration allows the slave to retrieve the function immediately from the function cache without the need to request the function definition from the master --- that requires (de-) serialization of the function.

\begin{listing}
\begin{javascriptcode*}{highlightlines={1, 3, 8, 16, 26-31}}
var imageWidth; |$\label{code:slave-image-width}$|

function computePixel(x, y) { |$\label{code:slave-compute-pixel-start}$|
	// ...
	return n;
}|$\label{code:slave-compute-pixel-end}$|

function computeMandelbrotLine(y) { |$\label{code:slave-task function-start}$|
	var line = new Uint8ClampedArray(imageWidth * 4);
	for (var x = 0; x < imageWidth; ++x) {
		line[x * 4] = computePixel(x, y);
	}
	return line;
}|$\label{code:slave-task function-end}$|
 
function _entrycomputeMandelbrotLine() { |$\label{code:slave-entry-function-start}$|
	try {
		var _environment = arguments[arguments.length - 1];
		imageWidth = _environment.imageWidth;
		return computeMandelbrotLine.apply(this, arguments);
	} finally {
		imageWidth = undefined;
	}
} |$\label{code:slave-entry-function-end}$|

slaveFunctionLookupTable.registerStaticFunction({|$\label{code:slave-register-function-start}$|
		identifier: 'static:_entrycomputeMandelbrotLine',
		_______isFunctionId: true
	}, _entrycomputeMandelbrotLine); |$\label{code:slave-register-function-end}$|
\end{javascriptcode*}
\caption{Generated Code by the Transpiler that is Executed on the Slaves}
\label{fig:transpiled-mandelbrot-slave}
\end{listing}

The transpiler further generates source maps that point back to the original location of the extracted task function and as well, transitive referenced functions. The source maps enable a true debugging experience allowing to set breakpoints inside of the browser developer tools\footnote{This is currently only supported by the developer tools of Google Chrome and Microsoft Edge.}. Without these source maps, breaking inside of a task function is only possible by using the inflexible \javascriptinline/debugger/ statement. The source maps further allow the browser to translate error messages back to the original code. This translation of the error messages helps to locate the root cause of errors from production more easily. The source map support is a distinct feature not offered by any of the related work.

\subsection{Implementation Restrictions}
The current implementation of the transpiler only supports the reactive API even though no technical reason therefore exists. Moreover, it only supports import statements according to the the ECMAScript 6 module specification~\cite[Section 15.2]{ecmaScript2015}.
