\section{Transpiler}\label{sec:transpiler}
The project further presents a JavaScript transpiler that hides some of the limitations for task-functions. It is based on top of WebPack\footnote{A JavaScript module bundler} and Babel\footnote{Framework for transforming JavaScriptCode}. The transpiler extracts the task-functions into the file loaded by the slaves and registers the function in the slaves function-cache, potentially resulting in a better runtime performance. The main advantage of using the transpiler over the bare runtime system is that task-functions are allowed to refer to variables --- including functions and imports --- from the outer scope and therefore reducing the programing model gap and allowing a far larger set of functions.


Furthermore, the transpiler adds debugging support for task-functions --- a distinct feature not offered by any related work. It generates source maps pointing back to the original source locations from where the task-functions have been extracted. Without these source maps, the only way to break inside of a task-function is by using the inflexible \javascriptinline/debugger/ statement. With the source maps pointing back to the original source break points can be set inside of the browsers developer tools giving a far better debugging experience to the developer\footnote{This is currently only supported by the developer tools of Google Chrome and Microsoft Edge}.

\subsection{Limitations}
The transpiler only uses static lexical scoping to resolve the binding of a variable and determining if a function is referenced. Therefore, the transpiler only supports references to functions where lexical scoping is sufficient to identify that the referenced value is a function. Furthermore, only ES2015 imports are supported. 
