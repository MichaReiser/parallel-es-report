\section{Conclusion}\label{sec:conclusion}
Multithreading is only sparsely used in nowedays JavaScript applications. Simply because the standards for creating background threads are platform-dependent and often enforce a messaging-based communication model. This paper presented a platform-independent type-safe API and runtime system that provides a seamless integration of parallel tasks into existing applications. It addresses the different needs of programmers by providing two APIs, a low-level API that allows running a single function in a background thread and a reactive API allowing parallelizing data-stream based operations with ease. 

The evaluation shows that most existing runtime systems perform similarly concerning execution time when applied to the given set of problems. Parallel.es is not always the fastest runtime system but never performs significantly slower. However, the main advantage of the library is its API and seamless integration into existing code. The proposed API is close to APIs widely used by the JavaScript community facilitating fast learning and straightforward transitioning of existing code. Moreover, Parallel.es is to be preferred for projects using typed language like TypeScript~\cite{typescript} or Flow~\cite{flow} since none of the related work offers a type-safe API. The presented transpiler has further the benefit that task functions can reference read-only variables and functions from its outer scope without any additional doing of the programmer. The related work on the contrary forces the programmer to structure the code in the way supported by the runtime system resulting in a clear break in the code style. The generated source maps help to identify errors from production and enable a pleasant debugging experience, a feature not offered by any related work. 

However, the evaluation also shows that the proposed system does not fit naturally with recursive problems like Quicksort or Knight-Tour that require a system supporting recursive tasks. Adding support for recursive tasks is non-trivial and subject of further research. Nevertheless, the proposed work eases writing multithreaded applications in JavaScript enabling them to efficiently use the hardware of the device.