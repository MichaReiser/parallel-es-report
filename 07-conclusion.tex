\section{Conclusion}\label{sec:conclusion}
A typical JavaScript application runs only inside of the main thread and therefore, only sparsely uses the available computation resources of nowadays computers. However, with the increasing complexity of the applications running in JavaScript, stagnating CPU frequencies, and the immersed rise of mobile and IoT devices the adoption of applications to multicore architecture is indispensable. Even though, writing multithreaded JavaScript applications is non-trivial because the standards for creating background threads are platform-dependent and often enforce a messaging-based communication model. This programming model gap creates a clearly visible seam between sequential and parallel code.

This paper presented a platform independent type-safe API and runtime system that provides a seamless integration of parallel tasks into existing applications. It addresses the different needs of programmers by providing two APIs, a low-level API that allows running a single function in a background thread and a reactive API allowing parallelizing data-stream based operations with ease. 

The evaluation shows that most existing runtime systems perform similarly concerning execution time when applied to the given set of problems. Parallel.es is not always the fastest runtime-system but never performs significantly slower than the others for any problem instance. However, the main advantage of the library is its API and seamless integration into existing code. The proposed API is close to APIs widely used by the JavaScript community facilitating fast learning and straightforward transitioning of existing code. Moreover, Parallel.es is to be preferred for projects using typed language like TypeScript~\cite{typescript} or Flow~\cite{flow} since none of the related work offers a type-safe API. The presented transpiler has further the benefit that task functions can reference read-only variables and functions from its outer scope without any additional doing of the programmer. The related work on the contrary forces the programmer to structure the code in the way supported by the runtime system resulting in a clear break in the code style. The additional transpilation step has further the advantage that the generated source maps help identifying errors from production and enable a pleasant debugging experience, a feature not offered by any related work. Nevertheless, the use of the transpiler remains optional for those preferring to avoid an additional build step.

Nevertheless, the evaluation also shows that the proposed system does not fit naturally with recursive problems like Quicksort or Knight-Tour that require a system supporting recursive tasks. Adding support for recursive tasks is non-trivial and subject of further research. Nevertheless, the proposed work eases creating multithreaded applications enabling them to use the computation power provided by the underlying hardware and is believed to contribute to the growth of multithreaded applications written in JavaScript. 