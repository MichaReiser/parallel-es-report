\section{Conclusion}\label{sec:conclusion}
Multithreading is only sparsely used in nowadays JavaScript applications because the standards for creating background threads are platform-dependent and often enforce a messaging-based communication model. This paper presented a platform-independent type-safe API and runtime system that provides a seamless integration of background tasks into existing applications. It addresses the different needs of programmers by providing two APIs, a low-level API for executing a single function in a background thread, and a reactive API allowing parallelizing data-stream-based operations with ease. 

The evaluation shows that most existing runtime systems perform similarly concerning execution time when applied to the given set of problems. Even though Parallel.es is one of the faster runtime systems, its main advantage is the API and the seamless integration into existing code. The proposed API is close to APIs widely used by the JavaScript community facilitating fast learning and straightforward transitioning of existing code. Moreover, the type-safety of the API is clearly beneficial for projects using typed languages like TypeScript~\cite{typescript} or Flow~\cite{flow}. The additional transpilation step allows a seamless integration of background tasks into existing code since a task function can reference read-only variables and functions from its outer scope without any additional doing of the programmer. The related work, on the contrary, defines more restrictions on task functions resulting in a clear seam between background tasks and the rest of the program. Besides the seamless integration, the additional transpilation step has the advantage that it generates source maps helping to identify errors from production and enabling a pleasant debugging experience, a feature not offered by any of the related work. 

However, the evaluation also shows that the proposed system does not fit naturally with recursive problems like Quicksort or Knight Tour that require a system supporting recursive tasks. Adding support for recursive tasks is non-trivial and subject of further research. Nevertheless, the proposed work eases writing multithreaded applications in JavaScript and enables them to use the device's hardware efficiently.